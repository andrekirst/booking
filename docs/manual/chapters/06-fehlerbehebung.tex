\section{Fehlerbehebung und FAQ}
\label{sec:fehlerbehebung}

\subsection{Häufige Probleme und Lösungen}

\subsubsection{Anmeldung funktioniert nicht}

\textbf{Problem:} Sie können sich nicht anmelden.

\textbf{Mögliche Ursachen und Lösungen:}

\begin{itemize}
    \item \textbf{Falsches Passwort}
        \begin{itemize}
            \item Überprüfen Sie Groß-/Kleinschreibung
            \item Nutzen Sie die „Passwort vergessen" Funktion
        \end{itemize}
    \item \textbf{Konto nicht freigeschaltet}
        \begin{itemize}
            \item Kontaktieren Sie einen Administrator
            \item Warten Sie auf die Freischaltung nach der Registrierung
        \end{itemize}
    \item \textbf{Browser-Probleme}
        \begin{itemize}
            \item Löschen Sie Browser-Cache und Cookies
            \item Versuchen Sie einen anderen Browser
            \item Deaktivieren Sie Werbeblocker temporär
        \end{itemize}
\end{itemize}

\subsubsection{Buchung kann nicht erstellt werden}

\textbf{Problem:} Das Buchungsformular zeigt Fehlermeldungen.

\textbf{Häufige Ursachen:}

\begin{itemize}
    \item \textbf{Ungültige Daten}
        \begin{itemize}
            \item Abreisedatum muss nach Anreisedatum liegen
            \item Nur zukünftige Daten sind erlaubt
            \item Maximale Buchungsdauer beachten
        \end{itemize}
    \item \textbf{Keine Verfügbarkeit}
        \begin{itemize}
            \item Gewählte Räume sind bereits belegt
            \item Versuchen Sie alternative Termine
            \item Prüfen Sie die Kalenderansicht
        \end{itemize}
    \item \textbf{Kapazität überschritten}
        \begin{itemize}
            \item Personenanzahl übersteigt Raumkapazität
            \item Wählen Sie zusätzliche Räume
        \end{itemize}
\end{itemize}

\subsubsection{Seite lädt nicht richtig}

\textbf{Problem:} Die Anwendung wird nicht korrekt angezeigt.

\textbf{Lösungsschritte:}

\begin{enumerate}
    \item Seite neu laden (F5 oder Strg+R)
    \item Browser-Cache leeren
    \item JavaScript aktivieren
    \item Browser aktualisieren
    \item Internet-Verbindung prüfen
\end{enumerate}

\subsection{Browser-Kompatibilität}

\subsubsection{Unterstützte Browser}

\begin{table}[H]
    \centering
    \begin{tabular}{|l|c|l|}
        \hline
        \textbf{Browser} & \textbf{Min. Version} & \textbf{Status} \\
        \hline
        Chrome & 90+ & \textcolor{green}{Vollständig unterstützt} \\
        Firefox & 88+ & \textcolor{green}{Vollständig unterstützt} \\
        Safari & 14+ & \textcolor{green}{Vollständig unterstützt} \\
        Edge & 90+ & \textcolor{green}{Vollständig unterstützt} \\
        Internet Explorer & Alle & \textcolor{red}{Nicht unterstützt} \\
        \hline
    \end{tabular}
    \caption{Browser-Kompatibilität}
    \label{tab:browser-compatibility}
\end{table}

\subsubsection{Mobile Browser}

\begin{itemize}
    \item iOS Safari (iOS 13+)
    \item Android Chrome (Version 90+)
    \item Samsung Internet Browser
    \item Firefox Mobile
\end{itemize}

\subsection{Fehlermeldungen verstehen}

\subsubsection{Validation-Fehler}

\begin{description}
    \item[„Startdatum muss in der Zukunft liegen"] Wählen Sie ein zukünftiges Datum
    \item[„Enddatum muss nach Startdatum liegen"] Korrigieren Sie die Datumsreihenfolge
    \item[„Kapazität überschritten"] Reduzieren Sie Personenanzahl oder wählen Sie weitere Räume
    \item[„Raum nicht verfügbar"] Wählen Sie alternative Daten oder andere Räume
\end{description}

\subsubsection{System-Fehler}

\begin{description}
    \item[„Verbindung zum Server fehlgeschlagen"] Prüfen Sie Ihre Internet-Verbindung
    \item[„Sitzung abgelaufen"] Melden Sie sich erneut an
    \item[„Unerwarteter Fehler"] Laden Sie die Seite neu und versuchen Sie es erneut
\end{description}

\subsection{Performance-Probleme}

\subsubsection{Langsame Ladezeiten}

\textbf{Mögliche Ursachen:}

\begin{itemize}
    \item Schwache Internet-Verbindung
    \item Viele geöffnete Browser-Tabs
    \item Veralteter Browser
    \item Viele Browser-Erweiterungen
\end{itemize}

\textbf{Lösungsansätze:}

\begin{itemize}
    \item Browser-Cache regelmäßig leeren
    \item Nicht benötigte Tabs schließen
    \item Browser aktualisieren
    \item Erweiterungen deaktivieren
\end{itemize}

\subsection{Häufig gestellte Fragen (FAQ)}

\subsubsection{Allgemeine Fragen}

\textbf{F: Kann ich meine Buchung nach der Bestätigung noch ändern?}
\textbf{A:} Bestätigte Buchungen können nur von Administratoren geändert werden. Kontaktieren Sie den Administrator für Änderungen.

\textbf{F: Wie lange im Voraus kann ich buchen?}
\textbf{A:} Buchungen sind bis zu 6 Monate im Voraus möglich.

\textbf{F: Kann ich für andere Personen buchen?}
\textbf{A:} Ja, Sie können Buchungen für Ihre Familie und Gäste erstellen. Geben Sie die entsprechende Personenanzahl an.

\textbf{F: Was passiert, wenn ich mein Passwort vergesse?}
\textbf{A:} Nutzen Sie die „Passwort vergessen" Funktion auf der Anmeldeseite. Sie erhalten eine E-Mail mit Reset-Link.

\subsubsection{Buchungsfragen}

\textbf{F: Bis wann kann ich kostenlos stornieren?}
\textbf{A:} Stornierungen sind bis 24 Stunden vor Anreise kostenlos möglich.

\textbf{F: Kann ich mehrere Räume gleichzeitig buchen?}
\textbf{A:} Ja, Sie können beliebig viele verfügbare Räume für denselben Zeitraum buchen.

\textbf{F: Gibt es eine maximale Aufenthaltsdauer?}
\textbf{A:} Die maximale Buchungsdauer beträgt 14 Nächte. Längere Aufenthalte nach Absprache mit Administratoren.

\subsection{Kontakt und Support}

\subsubsection{Wer hilft bei Problemen?}

\begin{description}
    \item[Technische Probleme] System-Administrator
    \item[Buchungsprobleme] Buchungs-Administrator
    \item[Account-Probleme] Benutzer-Administrator
    \item[Allgemeine Fragen] Dieses Handbuch oder kontextsensitive Hilfe
\end{description}

\subsubsection{Support-Kanäle}

\begin{itemize}
    \item \textbf{E-Mail}: support@buchungsplattform.local
    \item \textbf{Interne Hilfe}: Hilfe-Button in der Anwendung
    \item \textbf{Dokumentation}: Dieses Benutzerhandbuch
    \item \textbf{FAQ}: Häufige Fragen in der Anwendung
\end{itemize}

\subsubsection{Informationen für Support-Anfragen}

Geben Sie bei Support-Anfragen folgende Informationen an:

\begin{itemize}
    \item Ihr Benutzername/E-Mail
    \item Browser und Version
    \item Betriebssystem
    \item Genaue Fehlerbeschreibung
    \item Screenshots (falls möglich)
    \item Schritte zur Reproduktion des Problems
\end{itemize}

\newpage