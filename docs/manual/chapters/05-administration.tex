\section{Administration}
\label{sec:administration}

\textbf{Hinweis:} Dieses Kapitel richtet sich an Administratoren der Buchungsplattform. Normale Benutzer haben keinen Zugang zu diesen Funktionen.

\subsection{Administrator-Bereich aufrufen}

\subsubsection{Zugang zum Admin-Panel}

\begin{enumerate}
    \item Melden Sie sich mit einem Administrator-Konto an
    \item Klicken Sie auf \textbf{„Administration"} in der Navigation
    \item Der Administrator-Bereich wird geöffnet
\end{enumerate}

\begin{figure}[H]
    \centering
    \includegraphics[width=\textwidth]{images/admin-dashboard.png}
    \caption{Administrator-Dashboard mit Übersicht}
    \label{fig:admin-dashboard}
\end{figure}

\subsection{Benutzerverwaltung}

\subsubsection{Dashboard-Navigation}

Das Administrator-Dashboard ist in drei Hauptbereiche unterteilt:

\begin{description}
    \item[Allgemein] System-Übersicht und globale Einstellungen
    \item[Verwaltung] Benutzer-, Buchungs- und Raumverwaltung
    \item[Systemeinstellungen] E-Mail-Konfiguration und erweiterte Einstellungen
\end{description}

Die Navigation erfolgt über die vertikalen Tabs auf der linken Seite des Dashboards.

\subsubsection{Neue Benutzer freischalten}

Nach der E-Mail-Verifizierung müssen neue Registrierungen von Administratoren genehmigt werden:

\begin{enumerate}
    \item Navigieren Sie zu \textbf{„Verwaltung"} → \textbf{„Benutzer"}
    \item Neue Registrierungen werden mit Status \textbf{„Ausstehende Genehmigung"} angezeigt
    \item Der Verifizierungsstatus zeigt, ob die E-Mail-Adresse bestätigt wurde
    \item Überprüfen Sie die Benutzerangaben:
        \begin{itemize}
            \item Name und E-Mail-Adresse
            \item Registrierungsdatum
            \item E-Mail-Verifizierungsstatus
        \end{itemize}
    \item Klicken Sie auf \textbf{„Genehmigen"} oder \textbf{„Ablehnen"}
    \item Bei Ablehnung können Sie optional einen Grund angeben
    \item Der Benutzer erhält automatisch eine E-Mail-Benachrichtigung
\end{enumerate}

\begin{figure}[H]
    \centering
    \includegraphics[width=\textwidth]{images/user-approval.png}
    \caption{Freischaltung neuer Benutzerkonten mit E-Mail-Verifizierungsstatus}
    \label{fig:user-approval}
\end{figure}

\subsubsection{Benutzerübersicht}

Die Benutzerübersicht zeigt alle registrierten Benutzer mit folgenden Informationen:

\begin{itemize}
    \item \textbf{Name}: Vor- und Nachname des Benutzers
    \item \textbf{E-Mail}: E-Mail-Adresse und Verifizierungsstatus
    \item \textbf{Rolle}: Member, Administrator oder Inactive
    \item \textbf{Status}: Aktiv, Ausstehend oder Abgelehnt
    \item \textbf{Registrierung}: Datum der Registrierung
    \item \textbf{Aktionen}: Bearbeiten, Aktivieren/Deaktivieren
\end{itemize}

\subsubsection{Benutzerrollen verwalten}

\begin{description}
    \item[Member] Normale Familienmitglieder mit Buchungsrechten
    \item[Administrator] Vollzugriff auf alle Funktionen
    \item[Inactive] Deaktivierte Benutzer ohne Zugang
\end{description}

Rollen können über das Benutzer-Bearbeitungsformular geändert werden.

\subsection{Buchungsmanagement}

\subsubsection{Alle Buchungen verwalten}

Administratoren können alle Buchungen im System einsehen und verwalten:

\begin{itemize}
    \item \textbf{Buchungen bestätigen}: Pending-Buchungen genehmigen
    \item \textbf{Buchungen ablehnen}: Buchungen mit Begründung ablehnen
    \item \textbf{Buchungen bearbeiten}: Auch bestätigte Buchungen können geändert werden
    \item \textbf{Notfallstornierungen}: Buchungen in Ausnahmefällen stornieren
\end{itemize}

\begin{figure}[H]
    \centering
    \includegraphics[width=\textwidth]{images/admin-bookings.png}
    \caption{Administrator-Ansicht aller Buchungen}
    \label{fig:admin-bookings}
\end{figure}

\subsubsection{Buchungsrichtlinien durchsetzen}

\begin{itemize}
    \item Maximale Buchungsdauer überwachen
    \item Doppelbuchungen vermeiden
    \item Faire Verteilung der Buchungszeiten sicherstellen
    \item Besondere Anlässe berücksichtigen
\end{itemize}

\subsection{Raumverwaltung}

\subsubsection{Neue Räume hinzufügen}

\begin{enumerate}
    \item Navigieren Sie zu \textbf{„Raumverwaltung"}
    \item Klicken Sie auf \textbf{„Neuen Raum hinzufügen"}
    \item Füllen Sie die Raumdetails aus:
        \begin{itemize}
            \item Name des Raumes
            \item Typ (Zimmer, Schlafsaal, Camping, etc.)
            \item Maximale Kapazität
            \item Beschreibung und Ausstattung
        \end{itemize}
    \item Speichern Sie den neuen Raum
\end{enumerate}

\subsubsection{Bestehende Räume bearbeiten}

\begin{itemize}
    \item Raumdetails aktualisieren
    \item Kapazität anpassen
    \item Ausstattung ändern
    \item Räume temporär deaktivieren (z.B. bei Renovierung)
\end{itemize}

\subsubsection{Raum-Wartung und Sperrungen}

Für Wartungsarbeiten können Räume gesperrt werden:

\begin{enumerate}
    \item Wählen Sie den entsprechenden Raum
    \item Klicken Sie auf \textbf{„Wartung/Sperrung"}
    \item Geben Sie Zeitraum und Grund ein
    \item Der Raum wird für den Zeitraum nicht buchbar
\end{enumerate}

\subsection{System-Einstellungen}

\subsubsection{Globale Einstellungen}

\begin{itemize}
    \item \textbf{Maximale Buchungsdauer}: Standard-Limit für alle Benutzer
    \item \textbf{Vorlaufzeit}: Mindestzeit zwischen Buchung und Anreise
    \item \textbf{Stornierungsfrist}: Bis wann Buchungen storniert werden können
    \item \textbf{Wartungsmodus}: System für Wartungsarbeiten sperren
\end{itemize}

\subsubsection{E-Mail-Einstellungen}

Konfigurieren Sie die E-Mail-Einstellungen unter \textbf{„Systemeinstellungen"} → \textbf{„E-Mail"}:

\paragraph{SMTP-Server-Konfiguration}

\begin{itemize}
    \item \textbf{SMTP-Server}: Adresse Ihres E-Mail-Servers (z.B. smtp.gmail.com)
    \item \textbf{Port}: SMTP-Port (587 für TLS, 465 für SSL)
    \item \textbf{Benutzername}: Ihr E-Mail-Konto für den Versand
    \item \textbf{Passwort}: Das zugehörige Passwort
    \item \textbf{Verschlüsselung}: TLS oder SSL (empfohlen: TLS)
    \item \textbf{Absender-Name}: Angezeigter Name (z.B. „Garten-Buchungsplattform")
    \item \textbf{Absender-E-Mail}: E-Mail-Adresse des Absenders
\end{itemize}

\paragraph{E-Mail-Test}

Nach der Konfiguration:

\begin{enumerate}
    \item Geben Sie eine Test-E-Mail-Adresse ein
    \item Klicken Sie auf \textbf{„Test-E-Mail senden"}
    \item Überprüfen Sie den Posteingang der Test-Adresse
    \item Bei Erfolg: Speichern Sie die Einstellungen
    \item Bei Fehler: Überprüfen Sie die Konfiguration
\end{enumerate}

\paragraph{E-Mail-Benachrichtigungen}

Folgende automatische E-Mails werden versendet:

\begin{itemize}
    \item \textbf{Registrierungsbestätigung}: E-Mail-Verifizierung für neue Benutzer
    \item \textbf{Kontofreischaltung}: Benachrichtigung nach Admin-Genehmigung
    \item \textbf{Kontoablehnung}: Information bei Ablehnung mit Begründung
    \item \textbf{Buchungsbestätigungen}: Nach erfolgreicher Buchungserstellung
    \item \textbf{Buchungsstatus}: Bei Änderungen des Buchungsstatus
    \item \textbf{Erinnerungen}: Vor der Anreise (optional)
    \item \textbf{Admin-Benachrichtigungen}: Bei neuen Registrierungen und Buchungen
\end{itemize}

\subsection{Berichte und Statistiken}

\subsubsection{Buchungsstatistiken}

\begin{itemize}
    \item Auslastung nach Zeiträumen
    \item Beliebteste Räume
    \item Buchungsverhalten der Benutzer
    \item Stornierungsraten
\end{itemize}

\subsubsection{Exportfunktionen}

\begin{itemize}
    \item Buchungslisten als CSV exportieren
    \item Belegungspläne als PDF erstellen
    \item Statistiken für externe Auswertungen
\end{itemize}

\subsection{Backup und Wartung}

\subsubsection{Regelmäßige Aufgaben}

\begin{itemize}
    \item Veraltete Buchungen archivieren
    \item Benutzerkonten auf Aktualität prüfen
    \item System-Logs überwachen
    \item Performance-Metriken auswerten
\end{itemize}

\subsubsection{Notfall-Prozeduren}

\begin{itemize}
    \item System-Wiederherstellung aus Backup
    \item Notfall-Buchungen erstellen
    \item Benutzer-Support bei Login-Problemen
    \item Datenintegrität prüfen
\end{itemize}

\newpage