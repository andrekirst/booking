\section{Buchungen verwalten}
\label{sec:buchungen}

\subsection{Neue Buchung erstellen}

\subsubsection{Buchungsformular öffnen}

\begin{enumerate}
    \item Navigieren Sie zu \textbf{„Neue Buchung"} in der Hauptnavigation
    \item Das Buchungsformular wird geöffnet
\end{enumerate}

\begin{figure}[H]
    \centering
    \includegraphics[width=\textwidth]{images/booking-form.png}
    \caption{Buchungsformular für neue Buchungen}
    \label{fig:booking-form}
\end{figure}

\subsubsection{Datum und Zeitraum wählen}

\begin{enumerate}
    \item \textbf{Anreisedatum}: Klicken Sie auf das Kalender-Icon und wählen Sie das gewünschte Anreisedatum
    \item \textbf{Abreisedatum}: Wählen Sie das Abreisedatum (muss nach dem Anreisedatum liegen)
    \item Das System zeigt automatisch die Anzahl der Nächte an
    \item \textbf{Verfügbarkeit prüfen}: Das System prüft automatisch die Verfügbarkeit
\end{enumerate}

\textbf{Hinweise zur Datumswahl:}
\begin{itemize}
    \item Nur zukünftige Daten können gewählt werden
    \item Das Abreisedatum muss nach dem Anreisedatum liegen
    \item Buchungen sind für mindestens eine Nacht möglich
    \item Maximale Buchungsdauer: 14 Nächte
\end{itemize}

\subsubsection{Räume und Schlafplätze auswählen}

Nach der Datumswahl werden verfügbare Räume angezeigt:

\begin{figure}[H]
    \centering
    \includegraphics[width=\textwidth]{images/room-selection.png}
    \caption{Auswahl verfügbarer Räume und Schlafplätze}
    \label{fig:room-selection}
\end{figure}

Für jeden verfügbaren Raum können Sie:

\begin{itemize}
    \item \textbf{Raum auswählen}: Aktivieren Sie die Checkbox
    \item \textbf{Personenanzahl}: Geben Sie die Anzahl der Personen für diesen Raum an
    \item \textbf{Kapazität beachten}: Die maximale Kapazität wird angezeigt
    \item \textbf{Mehrere Räume}: Sie können mehrere Räume gleichzeitig buchen
\end{itemize}

\subsubsection{Zusätzliche Informationen}

\begin{description}
    \item[Notizen] Optionale Notizen für Ihre Buchung (z.B. besondere Wünsche, Anmerkungen)
    \item[Gesamtpersonen] Das System berechnet automatisch die Gesamtzahl der Personen
    \item[Verfügbarkeit] Rote Räume sind bereits belegt, grüne sind verfügbar
\end{description}

\subsection{Buchung bestätigen und absenden}

\begin{enumerate}
    \item Überprüfen Sie alle Angaben in der Zusammenfassung
    \item Klicken Sie auf \textbf{„Buchung erstellen"}
    \item Sie erhalten eine Bestätigung mit der Buchungsnummer
    \item Die Buchung hat zunächst den Status \textbf{„Pending"} (Wartend)
\end{enumerate}

\subsection{Buchungsübersicht}

\subsubsection{Ansichtsmodi}

Die Buchungsplattform bietet zwei verschiedene Ansichtsmodi für Ihre Buchungen:

\begin{description}
    \item[Listenansicht] Die traditionelle Ansicht zeigt alle Buchungen als Karten in einer Liste
    \item[Kalenderansicht] Die neue Kalenderansicht zeigt Buchungen in einem übersichtlichen Kalenderformat
\end{description}

Sie können zwischen beiden Ansichten mit den Schaltflächen rechts oben wechseln:
\begin{itemize}
    \item \textbf{Listen-Symbol}: Wechselt zur Listenansicht
    \item \textbf{Kalender-Symbol}: Wechselt zur Kalenderansicht
\end{itemize}

\subsubsection{Listenansicht}

Navigieren Sie zu \textbf{„Buchungen"} um alle Ihre Buchungen zu sehen:

\begin{figure}[H]
    \centering
    \includegraphics[width=\textwidth]{images/booking-list.png}
    \caption{Übersicht aller Buchungen in der Listenansicht}
    \label{fig:booking-list}
\end{figure}

In der Listenansicht sehen Sie:
\begin{itemize}
    \item Alle Ihre Buchungen als übersichtliche Karten
    \item Zeitraum und Status jeder Buchung
    \item Gebuchte Räume und Personenanzahl
    \item Aktionsbuttons zum Bearbeiten oder Stornieren
\end{itemize}

\subsubsection{Kalenderansicht}

Die Kalenderansicht bietet eine visuelle Übersicht über alle Buchungen:

\begin{figure}[H]
    \centering
    \includegraphics[width=\textwidth]{images/booking-calendar.png}
    \caption{Kalenderansicht mit Buchungsübersicht}
    \label{fig:booking-calendar}
\end{figure}

Funktionen der Kalenderansicht:
\begin{itemize}
    \item \textbf{Monatsansicht}: Standardmäßig wird der aktuelle Monat angezeigt
    \item \textbf{Navigation}: Mit den Pfeiltasten können Sie zwischen Monaten wechseln
    \item \textbf{Heute-Button}: Springt zum aktuellen Tag
    \item \textbf{Farbcodierung}: Verschiedene Farben für unterschiedliche Buchungsstatus
    \item \textbf{Details}: Klicken Sie auf eine Buchung für weitere Informationen
\end{itemize}

Auf der rechten Seite sehen Sie eine kompakte Liste aller Buchungen mit:
\begin{itemize}
    \item Zeitraum der Buchung
    \item Gebuchte Räume
    \item Anzahl der Personen
    \item Aktueller Status
\end{itemize}

\subsubsection{Buchungsstatus verstehen}

\begin{description}
    \item[Pending] \textcolor{orange}{Buchung wartet auf Administrator-Bestätigung}
    \item[Confirmed] \textcolor{green}{Buchung ist bestätigt und verbindlich}
    \item[Cancelled] \textcolor{red}{Buchung wurde storniert}
    \item[Rejected] \textcolor{red}{Buchung wurde vom Administrator abgelehnt}
\end{description}

\subsection{Buchung bearbeiten}

\subsubsection{Bestehende Buchung ändern}

Nur Buchungen mit Status \textbf{„Pending"} können bearbeitet werden:

\begin{enumerate}
    \item Klicken Sie auf das \textbf{Bearbeiten-Symbol} (Stift) bei der gewünschten Buchung
    \item Das Bearbeitungsformular öffnet sich mit den aktuellen Daten
    \item Nehmen Sie die gewünschten Änderungen vor:
        \begin{itemize}
            \item Daten ändern
            \item Räume hinzufügen oder entfernen
            \item Personenanzahl anpassen
            \item Notizen bearbeiten
        \end{itemize}
    \item Klicken Sie auf \textbf{„Änderungen speichern"}
\end{enumerate}

\textbf{Wichtig:} Bestätigte Buchungen können nicht mehr bearbeitet werden. Kontaktieren Sie in diesem Fall einen Administrator.

\subsection{Buchung stornieren}

\subsubsection{Stornierung durchführen}

\begin{enumerate}
    \item Klicken Sie auf das \textbf{Stornieren-Symbol} (X) bei der gewünschten Buchung
    \item Ein Bestätigungsdialog erscheint
    \item Bestätigen Sie die Stornierung
    \item Die Buchung erhält den Status \textbf{„Cancelled"}
\end{enumerate}

\begin{figure}[H]
    \centering
    \includegraphics[width=0.6\textwidth]{images/cancellation-dialog.png}
    \caption{Bestätigungsdialog für Buchungsstornierung}
    \label{fig:cancellation}
\end{figure}

\subsubsection{Stornierungsrichtlinien}

\begin{itemize}
    \item Buchungen können bis zu 24 Stunden vor Anreise storniert werden
    \item Stornierte Buchungen können nicht wiederhergestellt werden
    \item Bei kurzfristigen Stornierungen kontaktieren Sie einen Administrator
\end{itemize}

\subsection{Buchungsdetails anzeigen}

Klicken Sie auf eine Buchung, um detaillierte Informationen zu sehen:

\begin{itemize}
    \item Vollständige Buchungsdetails
    \item Historie aller Änderungen
    \item Kontaktinformationen
    \item Räume und Ausstattung
\end{itemize}

\newpage