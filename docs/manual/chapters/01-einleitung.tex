\section{Einleitung}
\label{sec:einleitung}

\subsection{Willkommen zur Buchungsplattform}

Die Buchungsplattform ist ein speziell entwickeltes System für Familienmitglieder, um Übernachtungen in einem gemeinsamen Garten bzw. Haus zu buchen und zu verwalten. Das System funktioniert ähnlich wie ein Hotel-Buchungssystem, ist aber auf die besonderen Bedürfnisse einer Familie zugeschnitten.

\subsection{Zielgruppe}

Diese Dokumentation richtet sich an:

\begin{itemize}
    \item \textbf{Familienmitglieder}: Personen, die Übernachtungen buchen möchten
    \item \textbf{Administratoren}: Personen, die das System verwalten und neue Mitglieder freischalten
    \item \textbf{System-Administratoren}: Technische Betreuer der Anwendung
\end{itemize}

\subsection{Funktionsübersicht}

Das System bietet folgende Hauptfunktionen:

\begin{description}
    \item[Buchungsmanagement] Erstellen, ändern und stornieren von Buchungen
    \item[Raumverwaltung] Verwaltung verfügbarer Schlafplätze und Räume
    \item[Benutzerverwaltung] Registrierung und Freischaltung von Familienmitgliedern
    \item[Kalenderansicht] Übersichtliche Darstellung aller Buchungen
    \item[Mobile Unterstützung] Optimiert für Smartphone und Tablet
\end{description}

\subsection{Systemvoraussetzungen}

\subsubsection{Für Benutzer}
\begin{itemize}
    \item Moderner Webbrowser (Chrome, Firefox, Safari, Edge)
    \item Internetverbindung
    \item Optional: Google-Account für vereinfachte Anmeldung
\end{itemize}

\subsubsection{Für Administratoren}
\begin{itemize}
    \item Alle Benutzervoraussetzungen
    \item Administrator-Berechtigung im System
    \item Zugang zur Admin-Oberfläche
\end{itemize}

\subsection{Sicherheit und Datenschutz}

\begin{itemize}
    \item Alle Daten werden verschlüsselt übertragen (HTTPS)
    \item Benutzerkonten sind passwortgeschützt
    \item Nur autorisierte Familienmitglieder haben Zugang
    \item Persönliche Daten werden nicht an Dritte weitergegeben
    \item Das System läuft auf einem privaten Server
\end{itemize}

\subsection{Hilfe und Support}

\begin{itemize}
    \item \textbf{Dieses Handbuch}: Umfassende Anleitung für alle Funktionen
    \item \textbf{Kontexthilfe}: Direkte Hilfe-Links in der Anwendung
    \item \textbf{FAQ}: Häufig gestellte Fragen im Anhang
    \item \textbf{Technischer Support}: Bei technischen Problemen wenden Sie sich an den System-Administrator
\end{itemize}

\newpage